\begin{itemize}
\item öffentlicher Schlüssel: $(n, e) = (299, 79)$.
\item geheimer Schlüsel: $d = 127$.
\end{itemize}
\begin{align*}
ed - 1 &= 79\times127 - 1 \\
&= 10033 - 1 \\
&= 10032 \\
&= 2^{s}\times t \\
&= 2^{4}\times 627
\end{align*}
Also $s = 4$ und $t = 627$. Für (2.1.26) müssen wir dann ein $a \in \{1,\dots,n-1\}$ finden mit
\begin{subequations}
\begin{align}
\mathrm{ggT}(a, n) &=\label{teilerfremd} 1 \\
\mathrm{ord}[a^{t}]_{p} &\neq\label{p-q-ordnung} \mathrm{ord}[a^{t}]_{q}
\end{align}
\end{subequations}
wobei wir $p$ und $q$ ja gerade noch nicht kennen. Wir ignorieren also die zweite Bedingung \eqref{p-q-ordnung} und gehen probabilistisch vor,
d.h. wir wählen ein zufälliges $a \in \{1,\dots,n-1\}$, welches \eqref{teilerfremd} erfüllt.
Mit den so bestimmten Variablen und einem $i \in \{0,\dots,s-1\} = \{0,1,2,3\}$
haben wir also $a, i, t, n$ sodass
\begin{align}
\mathrm{ggT}(a^{2^{i}t} - 1, n) \in \{p,q\}.
\end{align}

Zum Beispiel für $a = 224$, welches $\mathrm{ggT}(a,n) = 1$ erfüllt, ergibt sich
\begin{align*}
i = 0: \mathrm{ggT}(224^{2^{0}\times627} - 1, 299 ) = 13
\end{align*}
Also haben wir 13 als Teiler von 299 erkannt, und tatsächlich ist $299 / 13 = 23$.

Als Beispiel, dass auch $i > 0$ nötig sein kann, z.B. $a = 44$, dann ergibt sich
\begin{alignat*}{3}
i &= 0: \mathrm{ggT}(44^{2^{0}\times627} - 1, 299 ) &&= 1 \\
i &= 1: \mathrm{ggT}(44^{2^{1}\times627} - 1, 299 ) &&= 23
\end{alignat*}

Als Algorithmus nach (2.1.28) habe ich \texttt{Krypto/gap/RSAFactoring.g} geschrieben, dem man $n, e, d$ übergibt (der also bereits annimmt, dass man
$d$ aus $n$ und $e$ effizient errechnet hat) und der daraus einen Faktor von $n$ bestimmt.