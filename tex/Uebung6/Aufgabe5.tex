\begin{satz}[Lagrange]
Sei $G$ eine endliche Gruppe und $U \leq G$ eine Untergruppe von $G$. Dann gilt:
\begin{enumerate}
\renewcommand{\labelenumi}{(\theenumi)}
\item $\abs{U}$ teilt $\abs{G}$.
\item $\abs{_U \backslash ^G} = \abs{^G/_U} = \frac{\abs{G}}{\abs{U}}$.
\end{enumerate}
\end{satz}
\begin{proof}
Weil $G$ endlich ist, ist $U$ eine endliche Untergruppe von $G$. Setze $\abs{U} := m \in \mathbb{Z}_{\geq1}$. Zu jedem $g \in G$ sei
$gU := \{ gu : u \in U \}$ eine Linksnebenklasse von $U$.
Dann ist die Abbildung
\[
g\cdot : U \rightarrow gU,\, u \mapsto gu
\]
eine Bijektion: Weil $g \in G$ invertierbar ist, definiert $g^{-1}$ die Abbildung
\[
g^{-1}\cdot : gU \rightarrow U,\, gu \mapsto g^{-1}gu = u
\]
und es gilt
\[
g\cdot ( g^{-1}\cdot( gu ) ) = g\cdot ( u ) = gu
\]
sowie
\[
g^{-1}\cdot( g\cdot( u ) ) = g^{-1}\cdot( gu ) = u
\]

also $(g\cdot)\cdot(g^{-1}\cdot) = \mathrm{Id}_{gU}$ und $(g^{-1}\cdot)\cdot(g\cdot) = \mathrm{Id}_{U}$.
Also ist $g\cdot$ bijektiv. Insbesondere gilt $\abs{gU} = \abs{U}$, weil $U$ endlich ist.
Damit folgt $\abs{U} = \abs{gU} \forall g \in G$. 

Es sei $^G/_U := \{ gU : g \in G \}$ die Menge der Linksnebenklassen von $U$.
Für $g, h \in G$ untersuchen wir den Fall, dass $gU = hU$.
Unter der Äquivalenzrelation
\[
g \sim h :\Leftrightarrow gU = hU
\]
betrachte die Menge $^G/_\sim := \{ [g]_\sim : g \in G \}$ mit $[g]_\sim := \{ h \in G | h \sim g \}$. Offensichtlich gilt $^G/_U \simeq \,^G/_\sim$.
Im Fall $gU = hU$ gibt es also zu jedem $u_{1} \in U$ ein $u_{2} \in U$ sodass $gu_{1} = hu_{2}$, also $h^{-1}g = u_{1}^{-1}u_{2} \in U$. Damit
stellen wir fest, dass
\[
g \sim h \Leftrightarrow gU = hU \Leftrightarrow h^{-1}g \in U \Leftrightarrow g^{-1}h \in U.
\]


Die Anzahl der Nebenklassen von $U$ ist also ein Vielfaches von $\abs{U}$, d.h. $\exists n \in \mathbb{Z}_{\geq1} : \abs{^G/_U} = n\cdot \abs{U} = n\cdot m$.

\end{proof}
