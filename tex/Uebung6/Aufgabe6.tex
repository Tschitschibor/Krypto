Siehe \texttt{Krypto/gap/EllipticCurve.g}.

Zunächst habe ich eine Funktion \texttt{defines\_ellipse} geschrieben, die bei Eingabe von $a_{4}, a_{6}$ und dem Körper $F$ überprüft,
ob durch $E_{0}(a_{4}, a_,{6}, F)$ eine elliptische Kurve i.S.v. (2.3.10) definiert ist, wobei die Existenz des neutralen Elements ignoriert wird,
also nur die Eigenschaften in (2.3.6) überprüft werden, d.h.
\begin{align}
F \,\text{ist ein Körper}\\
\mathrm{Char}\,F \neq 2 \\
\mathrm{Char}\,F \neq 3 \\
4a_{4}^{3} + 27a_{6}^{2} \neq\label{istEllipse} 0 
\end{align}
wobei die letzte Gleichung über $F$ gesehen werden sollte.

Für die endlichen Körper benutze ich das gap-Paket \texttt{GaussForHomalg} vom homalg\_project (\url{https://github.com/homalg-project/homalg\_project}).

\begin{enumerate}[label=\alph*)]
\item $a_{4} = -7, a_{6} = -6, F = \mathbb{Z}_{5}, P = (3, 0)$

Zunächst betrachte ich die Zahlen als Elemente von $\mathbb{Z}_{5}$, d.h.
\begin{alignat*}{4}
a_{4} \mod 5 &= -7 &\mod 5 &= 3 \\
a_{6} \mod 5 &= -6 &\mod 5 &= 4 \\
P \mod 5 &= (3, 0) &\mod 5 &= (3, 0)
\end{alignat*}

In gap wird das durch Multiplizieren mit $1_{F}$ erledigt, d.h. \eqref{istEllipse} wird so überprüft:
\[
1_{F}*(4*a_{4}^{3} + 27*a_{6}^{2}) \neq 1_{F}*0
\]

Auch die Koeffizienten in \eqref{istEllipse} können wir über $\mathbb{Z}_{5}$ betrachten:
\begin{align*}
4 \mod 5 &= 4 \\
27 \mod 5 &= 2
\end{align*}

Nun ist aber $4\cdot3^{3} + 2\cdot4^{2} \mod 5 = 140 \mod 5 = 0$, also gilt \eqref{istEllipse} nicht, d.h. mit
$a_{4} = -7, a_{6} = -6, F = \mathbb{Z}_{5}$ handelt es sich nicht um eine Ellipse (was alle weiteren Rechnungen erübrigt).

Das ergebnis liefert auch unsere Funktion \texttt{defines\_ellipse}:
\begin{Verbatim}[commandchars=!@|,fontsize=\small,frame=single,label=Example]
  !gapprompt@gap>| !gapinput@Z5 := HomalgRingOfIntegers( 5 );|
  GF(5)
  !gapprompt@gap>| !gapinput@defines_ellipse( -7, -6, Z5 );|
  false
\end{Verbatim}

\item $a_{4} = -5, a_{6} = 5, F = \mathbb{Z}_{11}, P = (4, 7)$.

Über $\mathbb{Z}_{11}$ ergibt das
\begin{alignat*}{4}
a_{4} \mod 11 &= -5 &\mod 11 &= 6 \\
a_{6} \mod 11 &= 5 &\mod 11 &= 5 \\
P \mod 11 &= (4, 7) &\mod 11 &= (4, 7)
\end{alignat*}
Ebenso die Koeffizienten
\begin{align*}
4 \mod 11 &= 4 \\
27 \mod 11 &= 5
\end{align*}

Wir stellen fest, $\mathbb{Z}_{11}$ ist ein Körper, nicht von Charakteristik $2$ oder $3$. Bleibt also \eqref{istEllipse} zu überprüfen:
\begin{align*}
4\cdot 6^{3} + 5\cdot 5^{2} \mod 11 = 989 \mod 11 = 10 \neq 0 \mod 11.
\end{align*}

Das ergebnis liefert auch unsere Funktion \texttt{defines\_ellipse}:
\begin{Verbatim}[commandchars=!@|,fontsize=\small,frame=single,label=Example]
  !gapprompt@gap>| !gapinput@Z11 := HomalgRingOfIntegers( 11 );|
  GF(11)
  !gapprompt@gap>| !gapinput@defines_ellipse( -5, 5, Z11 );|
  true
\end{Verbatim}

Nun haben wir also eine elliptische Kurve $E_{0}(-5, 5, \mathbb{Z}_{11})$ und wollen die Anzahl der Elemente bestimmen.
Da sie als $x$-$y$-Graph eine Teilmenge der Zahlenebene $\mathbb{Z}_{11}^{2}$ ist, brauchen wir also nur für endlich viele
Punkte zu überprüfen, ob sie auf der elliptischen Kurve liegen. Dazu bestimmen wir zunächst die Gleichung der Kurve:
\begin{align}
y^2 =\label{Kurvengleichung} x^{3} + a_{4}x + a_{6},
\end{align}
also
\begin{align*}
y^{2} = x^{3} + 6x + 5
\end{align*}

Dazu habe ich eine Funktion \texttt{ellipse\_membership} geschrieben, die bei Eingabe eines Punktes $xy \in F^{2}$ und den
Werten $a_{4}, a_{6}$ und $F$ ausgibt, ob $xy$ die Gleichung \eqref{Kurvengleichung} erfüllt.

Diesen Filter kann ich nun auf die Zahlenebene $F^{2}$ anwenden, wenn $F$ endlich ist. Zusammen mit der Funktion \texttt{defines\_ellipse}
habe ich dann die Funktion \texttt{ellipticCurve} geschrieben, die zuerst überprüft, ob es sich um eine ellipse handelt, dann ob der Körper
endlich ist, und dann die Punkte auf der elliptischen Kurve als Teilmenge von $F^{2}$ ausgibt:

\begin{Verbatim}[commandchars=!@|,fontsize=\small,frame=single,label=Example]
  !gapprompt@gap>| !gapinput@E0 := ellipticCurve( -5, 5, Z11 );|
  [ [ 0*Z(11), Z(11)^2 ], [ 0*Z(11), Z(11)^7 ], [ Z(11)^0, Z(11)^0 ],
  [ Z(11)^0, Z(11)^5 ], [ Z(11), Z(11)^4 ], [ Z(11), Z(11)^9 ],
  [ Z(11)^2, Z(11)^2 ], [ Z(11)^2, Z(11)^7 ], [ Z(11)^3, Z(11) ],
  [ Z(11)^3, Z(11)^6 ], [ Z(11)^5, Z(11)^3 ], [ Z(11)^5, Z(11)^8 ],
  [ Z(11)^7, Z(11)^2 ], [ Z(11)^7, Z(11)^7 ], [ Z(11)^9, Z(11) ],
  [ Z(11)^9, Z(11)^6 ] ]
  !gapprompt@gap>| !gapinput@Length( E0 );|
  16
\end{Verbatim}
Wir dürfen aber auch nicht den Fernpunkt $O \in E_{0}$ vergessen, der das neutrale Element der Gruppe ist.
Die Gruppe hat also insgesamt 17 Elemente.

Damit können wir auf zwei Arten überprüfen, ob der Punkt $P = (4, 7)$ auf $E_{0}$ liegt:

\begin{Verbatim}[commandchars=!@|,fontsize=\small,frame=single,label=Example]
  !gapprompt@gap>| !gapinput@P := [ 4, 7 ];|
  [ 4, 7 ]
  !gapprompt@gap>| !gapinput@ellipse_membership( P, -5, 5, Z11 );|
  true
  !gapprompt@gap>| !gapinput@One( Z11 )*P in E0;|
  true
\end{Verbatim}

\item $a_{4} = -43, a_{6} = 166, F = \mathbb{R}, P = (3, 8)$.

Da $\mathbb{R}$ ein unendlicher Körper ist, gilt $\mathrm{Char}\,\mathbb{R} = 0$, und wir brauchen auch nicht in Restklassen zu rechnen.
Es gilt also wieder nur die Gleichung \eqref{istEllipse} zu überprüfen, also
\[
4\cdot (-43)^{3} + 27\cdot 166^{2} = 425984 \neq 0
\]
Statt über $\mathbb{R}$ rechnen wir in gap über den berechenbaren Körper $\mathbb{Q}$, was am Ergebnis nichts ändert:

\begin{Verbatim}[commandchars=!@|,fontsize=\small,frame=single,label=Example]
  !gapprompt@gap>| !gapinput@QQ := HomalgFieldOfRationals();|
  Q
  !gapprompt@gap>| !gapinput@defines_ellipse( -43, 166, QQ );|
  true
\end{Verbatim}

Da es sich bei $E_{0}( -43, 166, \mathbb{R}) \subset (\mathbb{R}^{2} \cup \{O\})$ um einen stetigen Graph handelt, ist die Gruppe
unendlich groß.

Um einen Punkt $P$ zu sich selbst zu addieren, also ihn zu verdoppeln in der Gruppen-Operation auf $E_{0}$ bildet man die Tangente an der
elliptischen Kurve durch $P$. Wenn die Tangente parallel zur $y$-Achse verläuft, schneidet sie die elliptische Kurve in keinem weiteren Punkt.
In diesem Fall gilt dann $P + P = O$ und die Ordnung von $P$ wäre zwei.

Wie man am Graph der Kurve
\begin{align*}
y^2 = x^{3} - 43x + 166,
\end{align*}
sehen kann, ist der äußerst linke Punkt derjenige, an dem $y = 0$ gilt und der eine Tangente parallel zur $y$-Achse hat.
Da es sich nicht um unseren Punkt $P = (3, 8)$ handelt, können wir also $P + P = O$ ausschließen, d.h. $P$ ist nicht von Ordnung 2.

Wir müssen also die Tangente an $E_{0}$ im Punkt $P = (3, 8)$ berechnen. Da sich der Punkt in der oberen Hälfte des Graphes befindet,
können wir $y > 0$ annehmen, und damit den Graph nach $y = f(x)$ auflösen:
\begin{alignat*}{3}
y &= f(x) &= \sqrt{x^{3} - 43x + 166} &= (x^{3} - 43x + 166)^{1/2}
\end{alignat*}
Für die Tangente bilden wir also die Ableitung von $f$:
\begin{alignat*}{2}
f'(x) &= (3x^{2} - 43)\cdot \frac{1}{2} \cdot (x^{3} - 43x + 166)^{-1/2} \\
&= \frac{1}{2} \cdot \frac{3x^{2}-43}{\sqrt{x^{3}-43x+166}}
\end{alignat*}
Die Tangente hat also im Punkt $P = (3,8)$ die Steigung $f'(3) = -1$. Damit können wir die Tangentenfunktion
$t_{P}(x)$ zur Geraden $L(P,P,\mathbb{R})$ bestimmen:
\begin{align*}
t_{P}(x) &= ax + b \\
t_{P}(3) &= 8 \\
t'_{P}(x) &= a = f'(3) = -1 \\
t_{P}(x) &= -x + b \\
8 &= -3 + b \\
b &= 11 \\
t_{P}(x) &= -x + 11
\end{align*}
Nun haben wir mit $P$ einen doppelten Schnittpunkt an $E_{0}$ und suchen den dritten Schnittpunkt $R \in L(P,P,\mathbb{R}) \cap E_{0}$,
also setzen wir beide Funktionen gleich und lösen nach $x$ und $y$ auf:
\begin{align*}
t_{P}(x) &= f(x) \\
-x + 11 &= \sqrt{x^{3} - 43x + 166} \\
(11 - x)^{2} &= x^{3} - 43x + 166 \\
121 - 22x + x^{2} &= x^{3} -43x + 166 \\
x^{3} - x^{2} - 21x + 45 &= 0 \\
(x^{3} - x^{2} - 21x + 45) : (x - 3) &= x^{2} + 2x - 15 \\
(x+5)(x-3)(x-3) &= 0 \\
\end{align*}
also $R = (-5, f(-5)) = (-5, 16)$.
Damit haben wir also $P\ast P$ nach (2.3.10) berechnet. Für die Gruppenoperation auf $E_{0}$ müssen wir dann nur noch berechnen
\begin{align*}
P + P &= 0 \ast (P \ast P) \\
&= 0 \ast R \\
&= 0 \ast (-5, 16) \\
&= (-5, -16)
\end{align*}
Es muss also nur noch das Vorzeichen vom $y$-Wert vertauscht werden. Wir halten also fest, $P + P = 2P = (-5, -16)$.
Nun sollte die Verbindungsline zwischen $P$ und $nP$ in den meisten Fällen keine Tangente an $E_{0}$ mehr sein, sondern eine Sekante.
Wir berechnen also der Reihe nach $3P = P + 2P, 4P = P + 3P,\dots$ bis wir für ein $n \in \mathbb{N}$ haben $nP = 0$.

\begin{align*}
L(P,Q,\mathbb{R}) &= \{(1-s)P + sQ \,|\,s \in \mathbb{R}\} \\
(x,y) &\in L(P,Q,\mathbb{R}) \\
\Leftrightarrow x &= (1-s)P_{x} + sQ_{x} \\
y &= (1-s)P_{y} + sQ_{y} \\
\end{align*}
Das ergibt für $L(P, 2P, \mathbb{R})$:
\begin{align*}
x &= (1-s)3 + s(-5) \\
y &= (1-s)8 + s(-16) \\
x(s) &= 3 -8s \\
y(s) &= 8 - 24s \\
s(x) &= (x-3)/(-8) \\
y(x) &= 8 - 24(s(x)) \\
&= 8 - 24( (x-3)/(-8) ) \\
&= 8 + 3(x-3) \\
&= 3x - 1
\end{align*}

Also wieder gleichsetzen
\begin{align*}
(3x - 1)^{2} = x^{3} - 43x + 166 \\
9x^{2} - 6x + 1 = x^{3} - 43x + 166 \\
x^{3} - 9x^{2} - 37x +165 = 0
\end{align*}
Und wir kennen bereits zwei Lösungen $P_{x} = 3$ und $(2P)_{x} = -5$, wir können also dividieren:
\begin{align*}
(x^{3} - 9x^{2} - 37x + 165):((x-3)(x+5)) = (x^{2} - 6x - 55):(x+5) = (x-11)
\end{align*}
Der dritte Punkt ist also $R = (11, f(11)) = (11, 32)$. Einmal noch das Vorzeichen vom $y$-Wert tauschen erhalten wir $3P = (11, -32)$.

Wieder die Gerade durch $P$ und $3P$:
\begin{align*}
x &= (1-s)3 + s(11) \\
&= 3 + 8s \\
s(x) &= (x-3)/8 \\
y &= (1-s)8 + s(-32) \\
y(x) &= 8 - 40( s(x) ) \\
&= 8 - 40( (x-3)/8 ) \\
&= 8 - 5x + 15 \\
&= -5x + 23
\end{align*}

Wieder gleichsetzen:
\begin{align*}
(-5x + 23)^{2} &= x^{3} - 43x + 166 \\
25x^{2} - 230x + 529 &= x^{3} - 43x + 166 \\
x^{3} - 25x^{2} + 187x - 363 &= 0
\end{align*}
Mit den bekannten zwei Lösungen $x = 3$ und $x = 11$:
\begin{align*}
(x^{3} - 25x^{2} + 187x - 363)&:((x-3)(x-11)) \\
= (x^{2} - 22x + 121)&:(x-11) \\
= x-11
\end{align*}

Wir haben also wieder $x = 11$ herausbekommen. Also $R = (11, f(11)) = (11, 32)$ also $P \ast 3P = 3P$.
Und $P + 3P = 0 \ast 3P = (11, -(-32)) = (11, 32) = 4P$.

Nun ist die Verbindungslinie zwischen $4P$ und $3P$ parallel zur $y$-Achse, also $4P \ast 3P = 0$ und damit
\begin{align}
4P + 3P = 0 \ast (4P \ast 3P) = 0 \ast 0 = 0 \\
\Rightarrow 7P = 4P + 3P = 0 \\
\Rightarrow \mathrm{ord}_{E_{0}} P = 7
\end{align}

Die Ordnung vom Punkt $P = (3, 8)$ beträgt $7$.

\end{enumerate}























