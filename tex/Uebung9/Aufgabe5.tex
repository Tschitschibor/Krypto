b)
\begin{align*}
p &= \left( p_{0}, p_{1}, p_{2}, p_{3} \right) \\
0 &\leq p_3 \leq p_2 \leq p_1 \leq p_0 \\
\sum_{i=1}^{4} p_i &= 1
\end{align*}

Einige Beispiele:
(i)
\[ p = (1, 0, 0, 0) \]
Dazu reicht ein Code $g : \Sigma \rightarrow \mathbb{Z}_{2}^{+}$, der dem Zeichen $0$ ein einzelnes Zeichen zuordnet,
und den restlichen drei Zeichen irgendwelche, beliebig lange, unterschiedliche Zeichen, z.B. der aus (a) mit Längenpolynom
$X + X^{2} + 2X^{3}$:

\begin{align*}
0 &\rightarrow 0 \\
1 &\rightarrow 11 \\
2 &\rightarrow 101 \\
3 &\rightarrow 100
\end{align*}

Die Länge von $g(1), g(2), g(3)$ spielt für den Erwartungswert keine Rolle, weil die Wahrscheinlichkeiten $0$ sind, und es gilt

\[
\mathrm{E}(|g(X_{p,1})|) = \abs{g(0)}\cdot 1 + \abs{g(1)}\cdot 0 + \abs{g(2)}\cdot 0 + \abs{g(3)}\cdot 0 = \abs{g(0)} = 1.
\]

(ii)
\[ p_{0} = \frac{1}{2} \]
Wir vergleichen den Erwartungswert $E(|h_{(1,1,2)}|)$ des optimalen Codes $h$ mit Längenvektor $(1,1,2)$ mit dem
Erwartungswert $E(|h'_{(0,4)}|)$ des optimalen Codes $h'$ mit Längenvektor $(0,4)$:

\begin{align*}
E(|h_{(1,1,2)}|) &= \frac{1}{2}\abs{h(0)} + p_{1}\abs{h(1)} + p_{2}\abs{h(2)} + p_{3}\abs{h(3)} \\
&= \frac{1}{2}\cdot 1 + p_{1}\cdot 2 + (p_{2} + p_{3})\cdot 3 \\
&\leq \frac{1}{2} + (p_{1} + p_{2} + p_{3})\cdot 3 \\
&= \frac{1}{2} + (1-\frac{1}{2})\cdot 3 \\
&= \frac{1}{2} + \frac{3}{2} = \frac{4}{2} \\
&= 2 = \frac{1}{2}\cdot 2 + (1 - \frac{1}{2})\cdot 2 = E(|h'_{(0,4)}|)
\end{align*}

Bei $p_{0} = \frac{1}{2}$ ist also $E(|h_{(1,1,2)}|) \leq E(|h'_{(0,4)}|)$, d.h. der Kode mit dem Längenpolynom $(0,1,1,2)$ ist sicher optimal.

(iii)
Im Fall von $p_{0} > \frac{1}{2}$ gilt $1-p_{0} < \frac{1}{2}$. Mit der Abschätzung $p_{1}\cdot 2 \leq p_{1}\cdot 3$ lösen wir nach $p_{0}$ auf:
\begin{align*}
p_{0} + (1-p_{0})\cdot 3 &< 2 \\
p_{0} + 3 - 3p_{0} &< 2 \\
-2p_{0} &< -1 \\
p_{0} &> \frac{1}{2}
\end{align*}

Also gilt $E(|h_{(1,1,2)}|) \leq E(|h'_{(0,4)}|)$ für alle $p_{0} \geq \frac{1}{2}$.

Wenn nun das erste Zeichen ``$0$'' weniger als die Hälfte auftritt (aber trotzdem am häufigsten), brauchen wir vielleicht weitere Kriterien
an $p_{1}, p_{2}, p_{3}$.

(iv)
\[ p_{1} \leq p_{0} < \frac{1}{2} \Rightarrow 1 - 2p_{0} > 0\]

Diesmal ohne die zu scharfe Abschätzung für $p_{1}$:
\begin{align*}
p_{0}\abs{h(0)} + p_{1}\abs{h(1)} + (1 - p_{0} - p_{1}) \cdot 3 < 2 \\
\Leftrightarrow 3 - 2p_{0} - p_{1} < 2 \\
\Leftrightarrow -2 p_{0} - p_{1} < -1 \\
\Leftrightarrow -p_{1} < -1 + 2p_{0} \\
\Leftrightarrow p_{1} > 1 - 2p_{0}
\end{align*}

Also haben wir für $p_{1}$ die neue Bedingung:
\[
p_{0} \geq p_{1} > 1 - 2p_{0} > 0
\]

D.h. der Kode mit Längenpolynom $(1,1,2)$ ist optimal falls $p_{0} \in \left[\frac{1}{2},1\right]$ oder
$p_{0} \in \left(0,\frac{1}{2}\right)$ und $p_{1} \in \left(1-2p_{0}, p_{0}\right]$.
Der Fall $p_{0} = \frac{1}{3} = p_{1}$ und damit $p_{2} + p_{3} = 1 -  p_{0} - p_{1} = \frac{1}{3}$ ergibt immer noch
\[
p_{0}\abs{h(0)} + p_{1}\abs{h(1)} + (1 - p_{0} - p_{1}) \\
= \frac{1}{3}\cdot 1 + \frac{1}{3}\cdot 2 + \frac{1}{3}\cdot 3 \\
= \frac{1}{3} + \frac{2}{3} + 1 = 2
\]
Für Werte $p_{0} < \frac{1}{3}$ und $p_{1} < \frac{1}{3}$ ist der Kode mit Längenpolynom $(1,1,2)$ nicht mehr optimal.
Das kann man auch an der Grafik unten ablesen.

(v)
\[ p_{3} \leq p_{2} \leq p_{1} < 1-2p_{0} > 0\]

Da $\abs{h(2)} = \abs{h(3)} = 3$, ist die genaue Verteilung zwischen $p_{2}$ und $p_{3}$ nicht wichtig. Es zählt nur
die Summe $p_{2} + p_{3}$, welche wir aber bereits aus $p_{2} + p_{3} = 1 - p_{0} - p_{1}$ kennen. Der Fall
\[ p_{1} < 1-2p_{0} \]
muss also nicht weiter behandelt werden: Hier gilt in jedem Fall, dass  $E(|h_{(1,1,2)}|) \geq E(|h'_{(0,4)}|)$, also der Kode
mit Längenpolynom $4X^2$ optimal ist.

Wir können die Bedingungen an $p_{0}$ und $p_{1}$ in ein Diagramm zeichnen und die Flächen markieren, in denen jeweils
der Kode mit Längenpolynom $X + X^2 + 2X^3$ oder der mit $4X^{2}$ optimal ist:\\
\[
\begin{tikzpicture}
  \begin{axis}[my style, 
  xtick={0, 0.3333, 0.5, 1}, xticklabels={$0$, $\frac{1}{3}$,$\frac{1}{2}$, $1$},
  ytick={0, 0.3333, 0.5, 1}, yticklabels={$0$, $\frac{1}{3}$,$\frac{1}{2}$, $1$},
  xmin=0, xmax=1, ymin=0, ymax=1]
    
    \addplot[name path=A, thick, domain=0:1, smooth]{x};
    \addplot[name path=C, thick, domain=0.3333:0.5, smooth]{1-2*x};
    \path[name path=D, thick] (0.5,0) -- (0.5,0.5);
    \path[name path=B] (\pgfkeysvalueof{/pgfplots/xmin},0) -- (\pgfkeysvalueof{/pgfplots/xmax},0);
    \path[name path=BB] (0.3,0) -- (0.5,0);
    \addplot [LightBlue] fill between [
        of=A and B,soft clip={domain=0.5:1},
    ];
    \addplot [LightBlue] fill between [
        of=A and C,soft clip={domain=0:0.5},
    ];
    \addplot [RoyalGreen] fill between [
        of=A and B,soft clip={domain=0:0.3333},
    ];
    \addplot [RoyalGreen] fill between [
        of=C and BB,
    ];
    \path (0.7,0.4) node{$X+X^{2}+2X^{3}$}
    (0.3,0.135) node{$4X^{2}$};
  \end{axis}
\end{tikzpicture}
\]










