\documentclass[a4paper,12pt,twoside]{article}

\usepackage[ 
  a4paper,
  footskip=0.7cm,
  margin=2.7cm,
  top=1.1cm,
  bottom=1.4cm
]{geometry}

\usepackage[german]{babel}

%%% Math theorem styles
\usepackage{amsthm}
\usepackage{amsmath}
\usepackage{amsfonts}
\usepackage{amsbsy}
\usepackage{amssymb}
\usepackage{mathtools}
\usepackage{esvect}
\usepackage{commath}
\usepackage[sc,osf]{mathpazo}

\theoremstyle{definition}
\newtheorem{satz}{Satz}[subsection]
\newtheorem{lemma}[satz]{Lemma}
\newtheorem{korollar}[satz]{Korollar}
\newtheorem{definition}[satz]{Definition}
\newtheorem{bemerkung}[satz]{Bemerkung}
\newtheorem{beispiel}[satz]{Beispiel}
\newtheorem{berechnung}[satz]{Berechnung}



\begin{document}

\section{Übung 1}
\section{Übung 2}
\section{Übung 3}
\section{Übung 4}
\newpage

\section{Übung 5}
\subsection{Aufgabe 1}
\subsection{Aufgabe 2}
\subsection{Aufgabe 3}
\subsection{Aufgabe 4}
\subsection{Aufgabe 5}
Siehe \texttt{Krypto/gap/ElGamal.g}
\newpage
\section{Übung 6}
\subsection{Aufgabe 1}
\subsection{Aufgabe 2}
\begin{itemize}
\item öffentlicher Schlüssel: $(n, e) = (299, 79)$.
\item geheimer Schlüsel: $d = 127$.
\end{itemize}
\begin{align*}
ed - 1 &= 79\times127 - 1 \\
&= 10033 - 1 \\
&= 10032 \\
&= 2^{s}\times t \\
&= 2^{4}\times 627
\end{align*}
Also $s = 4$ und $t = 627$. Für (2.1.26) müssen wir dann ein $a \in \{1,\dots,n-1\}$ finden mit
\begin{subequations}
\begin{align}
\mathrm{ggT}(a, n) &=\label{teilerfremd} 1 \\
\mathrm{ord}[a^{t}]_{p} &\neq\label{p-q-ordnung} \mathrm{ord}[a^{t}]_{q}
\end{align}
\end{subequations}
wobei wir $p$ und $q$ ja gerade noch nicht kennen. Wir ignorieren also die zweite Bedingung \eqref{p-q-ordnung} und gehen probabilistisch vor,
d.h. wir wählen ein zufälliges $a \in \{1,\dots,n-1\}$, welches \eqref{teilerfremd} erfüllt.
Mit den so bestimmten Variablen und einem $i \in \{0,\dots,s-1\} = \{0,1,2,3\}$
haben wir also $a, i, t, n$ sodass
\begin{align}
\mathrm{ggT}(a^{2^{i}t} - 1, n) \in \{p,q\}.
\end{align}

Zum Beispiel für $a = 224$, welches $\mathrm{ggT}(a,n) = 1$ erfüllt, ergibt sich
\begin{align*}
i = 0: \mathrm{ggT}(224^{2^{0}\times627} - 1, 299 ) = 13
\end{align*}
Also haben wir 13 als Teiler von 299 erkannt, und tatsächlich ist $299 / 13 = 23$.

Als Beispiel, dass auch $i > 0$ nötig sein kann, z.B. $a = 44$, dann ergibt sich
\begin{alignat*}{3}
i &= 0: \mathrm{ggT}(44^{2^{0}\times627} - 1, 299 ) &&= 1 \\
i &= 1: \mathrm{ggT}(44^{2^{1}\times627} - 1, 299 ) &&= 23
\end{alignat*}

Als Algorithmus nach (2.1.28) habe ich \texttt{Krypto/gap/RSAFactoring.g} geschrieben, dem man $n, e, d$ übergibt (der also bereits annimmt, dass man
$d$ aus $n$ und $e$ effizient errechnet hat) und der daraus einen Faktor von $n$ bestimmt.

\subsection{Aufgabe 3}
\subsection{Aufgabe 4}
Siehe \texttt{Krypto/gap/ElGamal.g}
\subsection{Aufgabe 5}
\begin{satz}[Lagrange]
Sei $G$ eine endliche Gruppe und $U \leq G$ eine Untergruppe von $G$. Dann gilt:
\begin{enumerate}
\renewcommand{\labelenumi}{(\theenumi)}
\item $\abs{U}$ teilt $\abs{G}$.
\item $\abs{_U \backslash ^G} = \abs{^G/_U} = \frac{\abs{G}}{\abs{U}}$.
\end{enumerate}
\end{satz}
\begin{proof}
Weil $G$ endlich ist, ist $U$ eine endliche Untergruppe von $G$. Setze $\abs{U} := m \in \mathbb{Z}_{\geq1}$. Zu jedem $g \in G$ sei
$gU := \{ gu : u \in U \}$ eine Linksnebenklasse von $U$.
Dann ist die Abbildung
\[
g\cdot : U \rightarrow gU,\, u \mapsto gu
\]
eine Bijektion: Weil $g \in G$ invertierbar ist, definiert $g^{-1}$ die Abbildung
\[
g^{-1}\cdot : gU \rightarrow U,\, gu \mapsto g^{-1}gu = u
\]
und es gilt
\[
g\cdot ( g^{-1}\cdot( gu ) ) = g\cdot ( u ) = gu
\]
sowie
\[
g^{-1}\cdot( g\cdot( u ) ) = g^{-1}\cdot( gu ) = u
\]

also $(g\cdot)\cdot(g^{-1}\cdot) = \mathrm{Id}_{gU}$ und $(g^{-1}\cdot)\cdot(g\cdot) = \mathrm{Id}_{U}$.
Also ist $g\cdot$ bijektiv. Insbesondere gilt $\abs{gU} = \abs{U}$, weil $U$ endlich ist.
Damit folgt $\abs{U} = \abs{gU} \forall g \in G$. 

Es sei $^G/_U := \{ gU : g \in G \}$ die Menge der Linksnebenklassen von $U$.
Für $g, h \in G$ untersuchen wir den Fall, dass $gU = hU$.
Unter der Äquivalenzrelation
\[
g \sim h :\Leftrightarrow gU = hU
\]
betrachte die Menge $^G/_\sim := \{ [g]_\sim : g \in G \}$ mit $[g]_\sim := \{ h \in G | h \sim g \}$. Offensichtlich gilt $^G/_U \simeq \,^G/_\sim$.
Im Fall $gU = hU$ gibt es also zu jedem $u_{1} \in U$ ein $u_{2} \in U$ sodass $gu_{1} = hu_{2}$, also $h^{-1}g = u_{1}^{-1}u_{2} \in U$. Damit
stellen wir fest, dass
\[
g \sim h \Leftrightarrow gU = hU \Leftrightarrow h^{-1}g \in U \Leftrightarrow g^{-1}h \in U.
\]


Die Anzahl der Nebenklassen von $U$ ist also ein Vielfaches von $\abs{U}$, d.h. $\exists n \in \mathbb{Z}_{\geq1} : \abs{^G/_U} = n\cdot \abs{U} = n\cdot m$.

\end{proof}

\subsection{Aufgabe 6}

\end{document}