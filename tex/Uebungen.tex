\documentclass[a4paper,12pt,twoside]{article}

\usepackage[ 
  a4paper,
  footskip=0.7cm,
  margin=2.7cm,
  top=1.1cm,
  bottom=1.4cm
]{geometry}

\usepackage[german]{babel}

%%% Math theorem styles
\usepackage{amsthm}
\usepackage{amsmath}
\usepackage{amsfonts}
\usepackage{amsbsy}
\usepackage{amssymb}
\usepackage{mathtools}
\usepackage{esvect}
\usepackage{commath}
\usepackage[sc,osf]{mathpazo}

\theoremstyle{definition}
\newtheorem{satz}{Satz}[subsection]
\newtheorem{lemma}[satz]{Lemma}
\newtheorem{korollar}[satz]{Korollar}
\newtheorem{definition}[satz]{Definition}
\newtheorem{bemerkung}[satz]{Bemerkung}
\newtheorem{beispiel}[satz]{Beispiel}
\newtheorem{berechnung}[satz]{Berechnung}



\begin{document}

\section{Übung 1}
\section{Übung 2}
\section{Übung 3}
\section{Übung 4}
\newpage

\section{Übung 5}
\subsection{Aufgabe 1}
\subsection{Aufgabe 2}
\subsection{Aufgabe 3}
\subsection{Aufgabe 4}
\subsection{Aufgabe 5}
Siehe \texttt{Krypto/gap/ElGamal.g}
\newpage
\section{Übung 6}
\subsection{Aufgabe 1}
\subsection{Aufgabe 2}
\subsection{Aufgabe 3}
\subsection{Aufgabe 4}
Siehe \texttt{Krypto/gap/ElGamal.g}
\subsection{Aufgabe 5}

\begin{satz}
Sei $G$ eine endliche Gruppe und $U \leq G$ eine Untergruppe von $G$. Dann ist $\abs{U}$ ein Teiler von $\abs{G}$.
\end{satz}
\begin{proof}
Sei $g \in G$. Dann gilt für die Menge $gU := \{gu\,|\,u \in U \}$.
\end{proof}

\subsection{Aufgabe 6}

\end{document}